% -*- LaTeX -*-
% $Id: Vision.tex,v 1.2 2002/11/04 13:38:30 ptr Exp $

\documentclass[a4paper,twoside]{article}

\usepackage{rup}

\title{Vision}
\project{Virtual Manual}
% \RevHistory{dd/mmmm/yyyy & x.x & & \cr \hline}
\begin{document}

\maketitle

\section{Introduction}

[The purpose of this document is to collect,
analyze, and define high-level needs and features of the 
``System Name''. It focuses on the capabilities needed by the
stakeholders, and the target users, and \textbf{why} these needs exist.
The details of how the 
``System Name'' fulfils these
needs are detailed in the use-case and supplementary specifications.]

[The introduction of the \textbf{Vision} document should provide
 an overview of the
entire document. It should include the purpose, scope, definitions, acronyms,
abbreviations, references, and overview of this \textbf{Vision} document.]

\subsection{Purpose}

[Specify the purpose of this \textbf{Vision} document.]

\subsection{Scope}

[A brief description of the scope of this \textbf{Vision}
document; what Project(s) it is associated with, and anything else that is
affected or influenced by this document.]

\subsection{Definitions, Acronyms and Abbreviations}

[This subsection should provide the
definitions of all terms, acronyms, and abbreviations required to properly
interpret the \textbf{Vision} document. This information may
be provided by
reference to the project Glossary.]

\subsection{References}

[This subsection should provide a complete
list of all documents referenced elsewhere in the \textbf{Vision}
document. Each document should be identified by title,
report number (if applicable), date, and publishing organization.
Specify the sources from which the
references can be obtained. This information may be provided by reference to an
appendix or to another document.]

\subsection{Overview}

[This subsection should describe what the
rest of the \textbf{Vision} document contains and explain how the document is
organized.]

\section{Positioning}

\subsection{Business Opportunity}

[Briefly describe the business opportunity being met by this
project.]

\subsection{Problem Statement}

[Provide a statement summarizing the problem being solved by
this project. The following format may be used:]

\begin{center}
\begin{minipage}[t]{2in}
\textit{The problem of}
\end{minipage}
\begin{minipage}[t]{4in}
[describe the problem]
\end{minipage}
\\[1em]
\begin{minipage}[t]{2in}
\textit{affects}
\end{minipage}
\begin{minipage}[t]{4in}
[the stakeholders affected by the problem]
\end{minipage}
\\[1em]
\begin{minipage}[t]{2in}
\textit{the impact of which is}
\end{minipage}
\begin{minipage}[t]{4in}
[what is the impact of the problem]
\end{minipage}
\\[1em]
\begin{minipage}[t]{2in}
\textit{a successful solution would be}
\end{minipage}
\begin{minipage}[t]{4in}
[list some key benefits of a successful solution]
\end{minipage}

\end{center}

\subsection{Product Position Statement}

[Provide an overall statement summarizing at the highest
level, the unique position the product intends to fill in the marketplace. The
following format may be used:]
\begin{center}
\begin{minipage}[t]{2in}
\textit{For}
\end{minipage}
\begin{minipage}[t]{4in}
[target customer]
\end{minipage}
\\[1em]
\begin{minipage}[t]{2in}
\textit{Who}
\end{minipage}
\begin{minipage}[t]{4in}
[statement of the need or opportunity]
\end{minipage}
\\[1em]
\begin{minipage}[t]{2in}
\textit{The (product name) is a}
\end{minipage}
\begin{minipage}[t]{4in}
[product category]
\end{minipage}
\\[1em]
\begin{minipage}[t]{2in}
\textit{That}
\end{minipage}
\begin{minipage}[t]{4in}
[statement of key benefit; that is---compelling reason to buy]
\end{minipage}
\\[1em]
\begin{minipage}[t]{2in}
\textit{Unlike}
\end{minipage}
\begin{minipage}[t]{4in}
[primary competitive alternative] (text manuals, electronic text-only
manuals, electonic 2D illustrated manuals with hyper references)
\end{minipage}
\\[1em]
\begin{minipage}[t]{2in}
\textit{Our product}
\end{minipage}
\begin{minipage}[t]{4in}
[statement of primary differentiation]
\end{minipage}
\end{center}

[A product position statement communicates the intent of the
application and the importance of the project to all concerned personnel.]

\section{Stakeholder and User Descriptions}

[To effectively
provide products and services that meet your stakeholders and users' real
needs, it is necessary to identify and involve all of the stakeholders as part
of the Requirements Modeling process.
You must also identify the users of the system and ensure that the
stakeholder community adequately represents them. This section provides a profile of the stakeholders and users
involved in the project and the key problems that they perceive to be addressed
by the proposed solution. It does not
describe their specific requests or requirements as these are captured in a
separate stakeholder requests artifact.
Instead it provides the background and justification for why the
requirements are needed.]

\subsection{Market Demographics}

[Summarize the key
market demographics that motivate your product decisions. Describe and position
target market segments. Estimate the market's size and growth by using the
number of potential users, or the amount of money your customers spend trying
to meet needs that your product or enhancement would fulfill. Review major industry
trends and technologies. Answer these strategic questions:

\begin{itemize}
  \item What is your organization's reputation
in these markets? 

  \item What would you like it to be?

  \item How does this product or service support your goals?
\end{itemize}

]

\subsection{Stakeholder Summary}

[Present a summary list of all the identified stakeholders.]

\begin{center}
\begin{minipage}[t]{1.2in}
\centering \textbf{Name}
\end{minipage}
\begin{minipage}[t]{2in}
\centering \textbf{Represents}
\end{minipage}
\begin{minipage}[t]{2in}
\centering \textbf{Role}
\end{minipage}\nopagebreak
\\[1em]
\begin{minipage}[t]{1.2in}
Name the stakeholder type.
\end{minipage}
\begin{minipage}[t]{2in}
Briefly describe what they represent with respect to the development.
\end{minipage}
\hskip 1ex
\begin{minipage}[t]{2in}
 [Briefly describe the role they are playing in the development.
  For example, Ensure this]
\end{minipage}
\end{center}

\subsection{User Summary}

[Present a summary list of all the identified users.]

\begin{center}
\begin{minipage}[t]{1.2in}
\centering\textbf{Name}
\end{minipage}
\begin{minipage}[t]{2in}
\centering\textbf{Description}
\end{minipage}
\hskip 1ex
\begin{minipage}[t]{2in}
\centering\textbf{Stakeholder}
\end{minipage}
\\[1em]
\begin{minipage}[t]{1.2in}
 Name the user type
\end{minipage}
\begin{minipage}[t]{2in}
[Briefly describe what they represent with respect to the system.]
\end{minipage}
\hskip 1ex
\begin{minipage}[t]{2in}
[List how the user is represented by the stakeholders.
 For example, Represented by Stakeholder 1.1]
\end{minipage}
\end{center}


\subsection{User Environment}

[Detail the working
environment of the target user. Here are some suggestions:
\begin{itemize}
  \item Number of people
involved in completing the task? Is this changing?
  \item How long is a task
        cycle? Amount of time spent in each activity? Is this changing?
  \item Any unique environmental constraints: mobile, outdoors,
        in-flight, etc.?
  \item Which systems platforms are in use today? Future platforms?
  \item What other applications are in use? Does your application
        need to integrate with them?
  \item This is where extracts from the Business Model could
        be included to outline the task and workers involved etc.]
\end{itemize}


\subsection{Stakeholder Profiles}

[Describe each
stakeholder in the system here by filling in the following table for each
stakeholder. Remember stakeholder types
can be as divergent as users, strategy departments and technical
developers. A thorough profile should
cover the following topics for each type of stakeholder:]

\subsubsection{``Stakeholder Name''}

\begin{center}
\begin{minipage}[t]{2in}
Representative
\end{minipage}
\begin{minipage}[t]{4in}
[Who is the stakeholder representative to the project? (optional if documented elsewhere.)
 What we want here is names.]
\end{minipage}
\\[1ex] %--------------------------------
\begin{minipage}[t]{2in}
Description
\end{minipage}
\begin{minipage}[t]{4in}
[Brief description of the stakeholder type.]
\end{minipage}
\\[1ex] %--------------------------------
\begin{minipage}[t]{2in}
Type
\end{minipage}
\begin{minipage}[t]{4in}
[Qualify the stakeholder's expertise, technical background, and degree of
 sophistication---that is, guru, business, expert, casual user, etc.]
\end{minipage}
\\[1ex] %--------------------------------
\begin{minipage}[t]{2in}
Responsibilities
\end{minipage}
\begin{minipage}[t]{4in}
[List the stakeholder's key responsibilities with regards to the system being
 developed that is, their interest as a stakeholder.]
\end{minipage}
\\[1ex] %--------------------------------
\begin{minipage}[t]{2in}
Success Criteria
\end{minipage}
\begin{minipage}[t]{4in}
[How does the stakeholder define success? How is the stakeholder rewarded?]
\end{minipage}
\\[1ex] %--------------------------------
\begin{minipage}[t]{2in}
Involvement
\end{minipage}
\begin{minipage}[t]{4in}
[How the stakeholder is involved in the project? Relate where possible
 to RUP workers that is, Requirements Reviewer etc.]
\end{minipage}
\\[1ex] %--------------------------------
\begin{minipage}[t]{2in}
Deliverables
\end{minipage}
\begin{minipage}[t]{4in}
[Are there any additional deliverables required by the stakeholder?
 These could be project deliverables or outputs from the system under development.]
\end{minipage}
\\[1ex] %--------------------------------
\begin{minipage}[t]{2in}
Comments / Issues
\end{minipage}
\begin{minipage}[t]{4in}
[Problems that interfere with success and any other relevant information go here.]
\end{minipage}
\end{center}

\subsection{User Profiles}

[Describe each
unique user of the system here by filling in the following table for each user
type. Remember user types can be as
divergent as gurus and novices. For example, a guru might need a sophisticated,
flexible tool with cross-platform support, while a novice might need a tool
that is easy to use and user-friendly. A thorough profile should cover the
following topics for each type of user:]</p>

\subsubsection{``User Name''}
\begin{center}
\begin{minipage}[t]{2in}
Representative
\end{minipage}
\begin{minipage}[t]{4in}
[Who is the user representative to the project? (optional if documented
 elsewhere.) This often refers to the Stakeholder that represents the set of users,
 for example, Stakeholder: Stakeholder1.]
\end{minipage}
\\[1em] % ----------------------------------------
\begin{minipage}[t]{2in}
Description
\end{minipage}
\begin{minipage}[t]{4in}
[A brief description of the user type.]
\end{minipage}
\\[1em] % ----------------------------------------
\begin{minipage}[t]{2in}
Type
\end{minipage}
\begin{minipage}[t]{4in}
[Qualify the user's expertise, technical background, and degree
of sophistication that is, guru, casual user, etc.]
\end{minipage}
\\[1em] % ----------------------------------------
\begin{minipage}[t]{2in}
Responsibilities
\end{minipage}
\begin{minipage}[t]{4in}
[List the user's key responsibilities with regards to the system
 being developed that is, captures details, produces
 reports, coordinates work, etc.]
\end{minipage}
\\[1em] % ----------------------------------------
\begin{minipage}[t]{2in}
Success Criteria
\end{minipage}
\begin{minipage}[t]{4in}
[How does the user define success? How is the user rewarded?]
\end{minipage}
\\[1em] % ----------------------------------------
\begin{minipage}[t]{2in}
Involvement
\end{minipage}
\begin{minipage}[t]{4in}
[How the user is involved in the project?
 Relate where possible to RUP workers that is,
 Requirements Reviewer, etc.]
\end{minipage}
\\[1em] % ----------------------------------------
\begin{minipage}[t]{2in}
Deliverables
\end{minipage}
\begin{minipage}[t]{4in}
[Are there any deliverables the user produces and, if so, for whom?]
\end{minipage}
\\[1em] % ----------------------------------------
\begin{minipage}[t]{2in}
Comments / Issues
\end{minipage}
\begin{minipage}[t]{4in}
[Problems that interfere with success and any other relevant
 information go here. These would include trends that make the
 user's job easier or harder.]
\end{minipage}
\end{center}

\subsection{Key Stakeholder / User Needs}

[List the key
problems with existing solutions as perceived by the stakeholder. Clarify the
following issues for each problem:

\begin{itemize}
  \item What are the reasons for this problem?
  \item How is it solved now?
  \item  What solutions does the stakeholder want?]
\end{itemize}

[It is important to
understand the \textbf{relative} importance the stakeholder or user places on
solving each problem. Ranking and cumulative voting techniques indicate
problems that \textbf{must} be solved versus issues they would
 like addressed.

Fill in the
following table --- if using ReqPro to capture the Needs, this could be an
extract or report from that tool.]

\begin{tabular}{c|c|c|c|c}
  Need &  Priority & Concerns & Current Solution & Proposed Solutions \\
  Broadcast messages \\
\end{tabular}


\subsection{Alternatives and Competition}

[Identify
alternatives the stakeholder perceives as available. These can include buying a
competitor's product, building a homegrown solution or simply maintaining the
status quo. List any known competitive choices that exist, or may become
available. Include the major strengths and weaknesses of each competitor as
perceived by the stakeholder.]

\subsubsection{``aCompetitor''}

\subsubsection{``anotherCompetitor''}

\section{Product Overview}

[This section provides a high level view of the product
capabilities, interfaces to other applications, and systems configurations.
This section usually consists of three subsections, as follows: 
\begin{itemize}
  \item Product perspective
  \item Product functions
  \item Assumptions and dependencies]
\end{itemize}

\subsection{Product Perspective}

[This subsection of the <b>Vision</b> document should put the
product in perspective to other related products and the user's environment. If
the product is independent and totally self-contained, state it here. If the
product is a component of a larger system, then this subsection should relate
how these systems interact and should identify the relevant interfaces between
the systems. One easy way to display the major components of the larger system,
interconnections, and external interfaces is via a block diagram.]

\subsection{Summary of Capabilities}

[Summarize the major benefits and features the product will
provide. For example, a <b>Vision</b> document for a customer support system
may use this part to address problem documentation, routing, and status
reporting without mentioning the amount of detail each of these functions requires.

Organize the functions so the list is understandable to the
customer or to anyone else reading the document for the first time. A simple
table listing the key benefits and their supporting features might suffice. For
example:]

\begin{center}
Customer Support System \\[2ex]
\begin{tabular}{p{2in}|p{3in}}
  Customer Benefit & Supporting Features \\
\hline
New support staff can quickly get up to speed. &
Knowledge base assists support personnel in quickly
identifying known fixes and workarounds \\
\hline
Customer satisfaction is improved because nothing falls through the cracks. &
Problems are uniquely itemized, classified and tracked throughout
the resolution process. Automatic notification occurs for any aging issues.\\
\hline
Management can identify problem areas and gauge staff workload.&
Trend and distribution reports allow high level review of problem status.\\
\hline
Distributed support teams can work together to solve problems. &
Replication server allows current database information to be shared
across the enterprise\\
\hline
Customers can help themselves, lowering support costs and improving
response time. &
Knowledge base can be made available over the Internet. Includes
hypertext search capabilities and graphical query engine
\end{tabular}
\end{center}


\subsection{Assumptions and Dependencies}

[List each of the factors that affects the features stated in
the <b>Vision</b> document. List assumptions that, if changed, will alter the 
\textbf{Vision}
document. For example, an assumption may state that a specific operating
system will be available for the hardware designated for the software product.
If the operating system is not available, the \textbf{Vision} document
 will need to change.]

\subsection{Cost and Pricing}

[For products sold to external customers and for many in-house
 applications, cost and pricing issues can directly impact the
applications definition and implementation. In this section, record any cost
and pricing constraints that are relevant. For example, distribution costs, (\#
of diskettes, \# CD-ROMs, CD mastering) or other cost of goods sold constraints
(manuals, packaging) may be material to the projects success, or irrelevant,
depending on the nature of the application.]

\subsection{Licensing and Installation}

[Licensing and installation issues can also directly impact
the development effort. For example, the need to support serializing, password
security or network licensing will create additional requirements of the system
that must be considered in the development effort.

Installation requirements may also affect coding, or create
the need for separate installation software.]

\section{Product Features}

[List and briefly describe the product features. Features are
the high-level capabilities of the system that are necessary to deliver
benefits to the users. Each feature is an externally desired service that
typically requires a series of inputs to achieve the desired result. For
example, a feature of a problem tracking system might be the ability to provide
trending reports. As the use-case model takes shape, update the description to
refer to the use cases.

Because the \textbf{Vision} document is reviewed by a wide
variety of involved personnel, the level of detail should be general enough for
everyone to understand. However, enough detail should be available to provide
the team with the information they need to create a use-case model.

To effectively manage application complexity, we recommend
for any new system, or an increment to an existing system, capabilities are
abstracted to a high enough level so 25--99 features result. These features
provide the fundamental basis for product definition, scope management, and
project management. Each feature will be expanded in greater detail in the use-case
model.

Throughout this section, each feature should be externally
perceivable by users, operators or other external systems. These features
should include a description of functionality and any relevant usability issues
that must be addressed. The following guidelines apply:

 Avoid design. Keep feature
descriptions at a general level. Focus on capabilities needed and why, (not
how)�  they should be implemented

 If you are using the Requisite
toolkit, all should be selected as requirements of type for easy reference and
tracking.]

\subsection{``aFeature''}

\subsection{``anotherFeature''}

\section{Constraints}

[Note any design constraints, external constraints or other
dependencies.]

\section{Quality Ranges}

[Define the quality ranges for performance, robustness, fault
tolerance, usability, and similar characteristics that are not captured in the
Feature Set.]

\section{Precedence and Priority}

[Define the priority of the different system features.]

\section{Other Product Requirements}

[At a high-level, list applicable standards, hardware or
platform requirements, performance requirements, and environmental
requirements.]

\subsection{Applicable Standards}

[List all standards with which the product must comply. These
can include legal and regulatory (FDA, UCC) communications standards (TCP/IP,
ISDN), platform compliance standards (Windows, Unix, etc.), and quality and
safety standards (UL, ISO, CMM).]

\subsection{System Requirements}

[Define any system requirements necessary to support the
application. These can include the supported host operating systems and network
platforms, configurations, memory, peripherals, and companion software.]

\subsection{Performance Requirements}

[Use this section to detail performance requirements.
Performance issues can include such items as user load factors, bandwidth or
communication capacity, throughput, accuracy, and reliability or response times
under a variety of loading conditions.]

\subsection{Environmental Requirements}

[Detail environmental requirements as needed. For hardware-
based systems, environmental issues can include temperature, shock, humidity,
radiation, etc. For software applications, environmental factors can include
usage conditions, user environment, resource availability, maintenance issues,
and error handling, and recovery.]

\section{Documentation Requirements}

[This section describes the documentation that must be
developed to support successful application deployment.]

\subsection{User Manual}

[Describe the purpose and contents of the User Manual.
Discuss desired length, level of detail, need for index, glossary of terms,
tutorial vs. reference manual strategy, etc. Formatting and printing constraints
should also be identified.]

\subsection{On-line Help}

[Many applications provide an on-line help system to assist
the user. The nature of these systems is unique to application development as
they combine aspects of programming (hyperlinks, etc) with aspects of technical
writing (organization, presentation). Many have found the development of
on-line help system is a project within a project that benefits from up-front
scope management and planning activity.]

\subsection{Installation Guides, Configuration, Read Me File}

[A document that includes installation instructions and
configuration guidelines is important to a full solution offering. Also, a Read
Me file is typically included as a standard component. The Read Me can include
a What's New With This Release section, and a discussion of
compatibility issues with earlier releases. Most users also appreciate
documentation defining any known bugs and workarounds in the Read Me file.]

\subsection{Labeling and Packaging}

[Today's state of the art applications provide a consistent
look and feel that begins with product packaging and manifests through
installation menus, splash screens, help systems, GUI dialogs, etc. This
section defines the needs and types of labeling to be incorporated into the
code. Examples include copyright and patent notices, corporate logos,
standardized icons and other graphic elements, etc.]

\section{Appendix 1 - Feature Attributes}

[Features should be given attributes that can be used to
evaluate, track, prioritize, and manage the product items proposed for
implementation. All requirement types and attributes should be outlined in the
Requirements Management Plan, however you may wish to list and briefly
describes the attributes for features that have been chosen. Following
subsections represent a set of suggested feature attributes.]

\subsection{Status}

[Set after negotiation and review by the project management
team. Tracks progress during definition of the project baseline.]

\begin{tabular}{l|p{4in}}
Proposed &
[Used to describe features that are under discussion but
  have not yet been reviewed and accepted by the ``official channel'',
  such as a working group consisting of representatives from the project team,
  product management and user or customer community.]\\
Approved &
[Capabilities that are deemed useful and feasible and have
  been approved for implementation by the official channel.] \\
Incorporated &
[Features incorporated into the product baseline at a specific point
 in time.]
\end{tabular}

\subsection{Benefit}

[Set by Marketing, the product manager or the business
analyst. All requirements are not created equal. Ranking requirements by their
relative benefit to the end user opens a dialogue with customers, analysts and
members of the development team. Used in managing scope and determining
development priority.]

\begin{tabular}{l|p{3in}}
Critical &
[Essential features. Failure to implement means the system
 will not meet customer needs. All critical features must be implemented in
  the release or the schedule will slip.]\\
Important &
[Features important to the effectiveness and efficiency of
  the system for most applications. The functionality cannot be easily provided
  in some other way. Lack of inclusion of an important feature may affect
  customer or user satisfaction, or even revenue, but release will not be
  delayed due to lack of any important feature.]\\
Useful & 
[Features that are useful in less typical applications, will
  be used less frequently, or for which reasonably efficient workarounds can be
  achieved. No significant revenue or customer satisfaction impact can be
  expected if such an item is not included in a release.]\\
\end{tabular}

\subsection{Effort}

[Set by the development team. Because some features require
more time and resources than others, estimating the number of team or
person-weeks, lines of code required or function points, for example, is the
best way to gauge complexity and set expectations of what can and cannot be
accomplished in a given time frame. Used in managing scope and determining
development priority.]

\subsection{Risk}

[Set by development team based on the probability the project
will experience undesirable events, such as cost overruns, schedule delays or
even cancellation. Most project managers find categorizing risks as high,
medium, and low sufficient, although finer gradations are possible. Risk can
often be assessed indirectly by measuring the uncertainty (range) of the
projects teams schedule estimate.]

\subsection{Stability}

[Set by analyst and development team based on the probability
the feature will change or the team's understanding of the feature will change.
Used to help establish development priorities and determine those items for
which additional elicitation is the appropriate next action.]

\subsection{Target Release}

[Records the intended product version in which the feature
will first appear. This field can be used to allocate features from a
\textbf{Vision} document into a particular baseline release.
 When combined with the status
field, your team can propose, record and discuss various features of the
release without committing them to development. Only features whose Status is
set to Incorporated and whose Target Release is defined will be implemented.
When scope management occurs, the Target Release Version Number can be
increased so the item will remain in the \textbf{Vision} document but will be
scheduled for a later release.]

\subsection{Assigned To}

[In many projects, features will be assigned to ``feature
teams'' responsible for further elicitation, writing the software
requirements and implementation. This simple pull down list will help everyone
on the project team better understand responsibilities.]

\subsection{Reason}

[This text field is used to track the source of the requested
feature. Requirements exist for specific reasons. This field records an
explanation or a reference to an explanation. For example, the reference might
be to a page and line number of a product requirement specification, or to a
minute marker on a video of an important customer interview.]


\end{document}
