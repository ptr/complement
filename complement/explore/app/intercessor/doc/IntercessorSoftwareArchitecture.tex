% -*- LaTeX -*-

\documentclass[a4paper,twoside]{article}

% \usepackage[koi8-r]{inputenc}
\usepackage[T1]{fontenc}
\usepackage[pdftex]{graphics}
\usepackage{pslatex}
% \usepackage[dvips,final]{epsfig}
% \usepackage{cm-super}
\usepackage[colorlinks,urlcolor=blue]{hyperref}
\usepackage{url}
\usepackage{rup}

%% Define a new 'leo' style for the package that will use a smaller font.
\makeatletter
\def\url@leostyle{%
  \@ifundefined{selectfont}{\def\UrlFont{\sf}}{\def\UrlFont{\small\ttfamily}}}
\makeatother
%% Now actually use the newly defined style.
\urlstyle{leo}

\version{1.2}
\title{Software Architecture Document}

\project{Intercessor}

\RevHistory{%
19 Feb 2007 & 1.0 & Initial document & Petr Ovtchenkov \cr \hline
05 Mar 2007 & 1.1 & Functionality changes: early error processing added & Petr Ovtchenkov \cr \hline
07 Mar 2007 & 1.2 & Detailed error codes & Petr Ovtchenkov \cr \hline
}

\newcommand{\Inter}{{\fontseries{b}\selectfont ``Intercessor''}}

\begin{document}

\maketitle

% [Note: The following template is provided for use with the
% Rational Unified Process. Text enclosed
% in square brackets and displayed in blue italics is included
% to provide guidance to the author and should be deleted before publishing the
% document. A paragraph entered following this style will automatically be set to
% normal.]

% toc
\tableofcontents

\section{Introduction}

% [The introduction of the Software Architecture
% Document should provide an overview of the entire Software
% Architecture Document. It should include the purpose, scope,
% definitions, acronyms, abbreviations, references, and overview of the Software
% Architecture Document.]

\subsection{Purpose}

This document provides a comprehensive architectural overview of the \Inter{} server, using a
number of different architectural views to depict different aspects of the
\Inter{}. It is intended to capture and
convey the significant architectural decisions which have been made on the
system.

\subsection{Scope}

Software Architecture Document applies to \Inter{} proxy server,
middleware that help
to publish reports related to some server via Web service.

\subsection{Definitions, Acronyms and Abbreviations}

% [This subsection should provide the definitions of all terms,
% acronyms, and abbreviations required to properly interpret the Software
% Architecture Document. This information may be provided by
% reference to the project Glossary.]
\begin{description}
  \item[HTTP] Hypertext Transfer Protocol
  \item[TCP/IP] Transmission Control Protocol and the Internet Protocol (the set of network communications protocols)
  \item[URL] Universal Resource Locator
  \item[LWP] LightWeight Process
  \item[SMP] Symmetrical MultiProcessing
  \item[OE] Operational Environment
  \item[EDA] Event-Driven Architecture
  \item[POSIX] Portable Operating System Interface
  \item[BSD] Berkeley Software Distribution
  \item[pid] Process ID
\end{description}

\subsection{References}

\bibliographystyle{plain}
\bibliography{Bib}

\subsection{Overview}

% [This subsection should describe what the rest of the Software
% Architecture Document contains and explain how the Software
% Architecture Document is organized.]

Following Software Architecture Document consider \Inter{} design and usage from various
points of view (sections~\ref{ArchRep}--\ref{DataView}), performance (section~\ref{SizePerf}) and
aspects of self-testing and quality of \Inter{} (section~\ref{Quality}).

\section{Architectural Representation\label{ArchRep}}

% [This section describes what software architecture is for the
% current system, and how it is represented. Of the Use-Case, Logical,
% Process, Deployment, and Implementation
% Views, it enumerates the views that are necessary, and for each view,
% explains what types of model elements it contains.]

Architectural Representation illustrates \Inter{} design from various aspects:
goals (section~\ref{ArchGoals}), usage (section~\ref{UCView}), logical decomposition (section~\ref{LogicalView}),
concurrency (section~\ref{ProcessView}), deployment (section~\ref{DeplView}), implementation (section~\ref{ImplView}).

Software Architecture Document contains description of project goals and constraints (section~\ref{ArchGoals}).

Section~\ref{UCView} shows intended usage of this software and explains why this software
needed.

Section~\ref{LogicalView} demonstrates the logical position of this software and
interaction with neighbours.

Section~\ref{ProcessView} focuses on the processes and LWP, this may give useful info for running
this software on SMP OE and estimations of performance scaling of this software.

Section~\ref{DeplView} describes what shipped within this software, OE requirements
for proper functionality and how to integrate this software into production.

Section~\ref{ImplView} demonstrates how \Inter{} was implemented. It includes horizontal
and vertical decomposition.

\section{Architectural Goals and Constraints\label{ArchGoals}}

Request from some server Web interface to some report generation server
may take a some (unpredictable) time. Because of time restrictions for answer
on Web server side, we need to complete request in a short time, and allow
to read complete result (when it will be ready) in another request.

There are unconfirmed information, that ReportWare Server accept only \verb|POST| requests.

It is known, that body of request MUST be XML request (see section~\ref{DataView} for
request sample), otherwise ReportWare Server drop connection without any response.

Some problems (like wrong ReportWare server URL, bad request content, ans so on) may produce empty response or quick negative response.
It will be good if script on Web Server side receive negative response in such
cases immediately.

If the error is reported by ReportWare Server, \Inter{} should immediately return
all the ReportWare Server response in addition to error code.
 


% [This section describes the software requirements and
% objectives that have some significant impact on the architecture, for example,
% safety, security, privacy, use of an off-the-shelf product, portability,
% distribution, and reuse. It also captures the special constraints that may
% apply: design and implementation strategy, development tools, team structure,
% schedule, legacy code, and so on.]

\section{Use-Case View\label{UCView}}

Client makes request for report generation. Because report generation takes
some time, and connection between client and web server may be lost during
report generation, client sees something like ``Waiting...'' and then, after
some time, browser refreshes page and user will see report (or error message).

\subsection{Use-Case Realizations}

%[This section illustrates how the software actually works by
%giving a few selected use-case (or scenario) realizations, and explains how the
%various design model elements contribute to their functionality.]

Script running within scope of Web server forward request to \Inter.
\Inter{} validate request and re-translate request to ReportWare server.
If request from script was ill-formed, \Inter{} give negative response to script;
otherwise it gives positive response.

\Inter{} asynchronously transmit request to ReportWare Server, read response and
save content of the response to a file.

Script, running within scope of Web server, may check file with ReportWare response
some time later and deliver content (may be after additional processing) to user,
within separate HTTP-request from Web browser to Web Server.

\section{Logical View\label{LogicalView}}

% [This section describes the architecturally significant parts
% of the design model, such as its decomposition into subsystems and packages.
% And for each significant package, its decomposition into classes and class
% utilities. You should introduce architecturally significant classes and
% describe their responsibilities, as well as a few very important relationships,
% operations, and attributes.]

\subsection{Overview}

The \Inter{} looks like HTTP proxy, but has significant difference: content
delivery isn't transparent for script.

\Inter{} accept request from script, validate this request, and forward request
(as event) to object responsible for communication with ReportWare Server (fig.~\ref{IntercessorObjDiagram}).
This object check 
the ability to perform request and sends it to ReportWare Server. If this object
see positive HTTP response from ReportWare Server (i.e. request was accepted by ReportWare Server)
within time limit (3 seconds), it notify script about success (fig.~\ref{IntercessorTimeDiagram}); otherwise it
notify script about request failure (fig.~\ref{IntercessorTimeDiagramNeg}).

The content, received from
ReportWare Server, will be stored in special catalog under name, was specified in initial
request (fig.~\ref{IntercessorObjDiagram}, \ref{IntercessorTimeDiagram}).

% [This subsection describes the overall decomposition of the
% design model in terms of its package hierarchy and layers.]

\begin{figure}
\begin{center}
%  \input intercessor-obj.pstex_t
% \resizebox{3in}{!}{\includegraphics{intercessor-obj.pdf}}
\scalebox{0.8}{\includegraphics{intercessor-obj.pdf}}
\end{center}
\caption{Position of \Inter{}, data flow and objects interaction diagram.\label{IntercessorObjDiagram}}
\end{figure}

\begin{figure}
\begin{center}
%  \input intercessor-time.pstex_t
\scalebox{0.8}{\includegraphics{intercessor-time.pdf}}
\end{center}
\caption{Position of \Inter{}, time diagram.\label{IntercessorTimeDiagram}}
\end{figure}

\begin{figure}
\begin{center}
\scalebox{0.8}{\includegraphics{intercessor-time-neg.pdf}}
\end{center}
\caption{Position of \Inter{}, time diagram. Behaviour when response from ReportWare Server come beyond expected time interval.\label{IntercessorTimeDiagramNeg}}
\end{figure}

\subsection{Architecturally Significant Design Packages}

\subsubsection{``Boost'' project, filesystem}

``Filesystem'' (\cite{BoostProject:filesystem}) provide facilities to query and manipulate paths, files, and directories.

\subsubsection{``Boost'' project, program options}

``Program Options'' (\cite{BoostProject:programoptions}) provide convenient
and uniform way for command line options processing.

\subsubsection{``Boost'' project, regex}

``RegEx'' (\cite{BoostProject:regex}) implements fast and rich regular expressions
implementation; it used in HTTP protocol implementation.

\subsubsection{``Boost'' project, unit test framework}

``Unit Test Framework'' (\cite{BoostProject:testutf}) used for unit test of \Inter{}.

\subsubsection{STLport}

STLport (\cite{STLportProject}) used for effective STL implementation
and important extentions to base STL (access to file descriptor in fstreams---useful utilities from ``Complement'' project \cite{ComplementProject}).

\subsubsection{``Complement'' project}

``Complement'' project (\cite{ComplementProject}) provide high level of abstraction for multi-thread programming,
IPC, work with BSD sockets and EDA paradigm implementation.


%
%
%[For each significant package, include a subsection with its
%name, its brief description, and a diagram with all significant classes and
%packages contained within the package.
%
%For each significant class in the package, include its name,
%brief description, and, optionally a description of some of its major
%responsibilities, operations and attributes.]

\section{Process View\label{ProcessView}}

% [This section describes the system's decomposition into
% lightweight processes (single threads of control) and heavyweight processes
% (groupings of lightweight processes). Organize the section by groups of
% processes that communicate or interact. Describe the main modes of
% communication between processes, such as message passing, interrupts, and
% rendezvous.]

\Inter{} was designed as multi-threaded server. The number of threads is restricted
in the connection processing server (\verb|sockios|, \cite{ComplementProject}) and
depends upon intensity of requests to server. Default threads limit is 32 (for HTTP-server
part of \Inter{}).

The ``client'' part of \Inter{} is based on event-driven architecture (\verb|stem|, \cite{ComplementProject}).
It take at least 2 threads.

\Inter{} may be run as daemon (\verb|-d| or \verb|--daemon| option for \verb|intercessor|).

\section{Deployment View\label{DeplView}}

\subsection{Repository}

Commit project from SVN repository.

\subsection{Build}

\begin{verbatim}
(cd intercessor; ./configure; make)
\end{verbatim}

\subsection{Check build\label{DeplCheck}}

Run unit tests of all components of \Inter{}:
\begin{verbatim}
(cd intercessor; make check 2>&1) | tee int.log
grep -ie "\(error\)\|\(fail\)" int.log | grep -v exec_fail
\end{verbatim}
Check that result contain only ``\verb|*** No errors detected|'' (or ``\verb|**** no errors detected|'') records.

For deployment you will need:
\begin{itemize}
  \item binary \verb|intercessor/build/bin/intercessor|
  \item shared libraries listed as result of output
        \verb+v=`ldd intercessor/build/bin/intercessor | \+ \\
        \verb+grep intercessor/build/lib | \+ \\ 
        \verb+awk '{ print $3  "*"; }'`; for f in $v; do echo $f; done+ ' % $
%  \item files listed as result of output \verb=ldd intercessor/build/bin/intercessor | \= \\
%   \verb=grep intercessor/build/lib | \= \\
%   \verb=awk '{ print $3  "*"; }' | xargs /bin/sh -c 'ls $@'=
\end{itemize}

\subsection{Deployment}

\begin{figure}
\begin{center}
%  \input intercessor-depl.pstex_t
\scalebox{0.8}{\includegraphics{intercessor-depl.pdf}}
\end{center}
\caption{\Inter{} deployment.\label{IntercessorDeplDiagram}}
\end{figure}

\Inter{} should be installed so it can share same filesystem as scripts
running in the Web Server space (see fig.~\ref{IntercessorDeplDiagram}). The place for stored files should
be accessible from scripts running in the Web Server space. By security
reasons, this catalog shouldn't be used for other purposes.

\Inter{} should has ability to connect to ReportWare Server via TCP/IP connection.

\Inter{} should has ability to listen one TCP port.

Libraries that shipped with \Inter{} should be searchable by loader:
ones should be situated either in catalogs mentioned in \verb|/etc/ld.so.conf|
or in catalogs from \verb|LD_LIBRARY_PATH| environment variable. You can check
that all libraries accessible with \verb|ldd intercessor|. The dynamic libraries,
that was built within \Inter{} project are:
\begin{itemize}
  \item \verb|libxmt.so.1.10|
  \item \verb|libsockios.so.1.11|
  \item \verb|libstem.so.4.4|
  \item \verb|libboost_regex.so.1.33|
  \item \verb|libboost_fs.so.1.33|
  \item \verb|libboost_program_options.so.1.33|
  \item \verb|libstlport.so.5.2|
\end{itemize}

Unit test (\verb|ut_intercessor|) required additional libraries:
\begin{itemize}
  \item \verb|libboost_test_utf.so.1.33|
\end{itemize}

Available options for \Inter{}:
\begin{itemize}
  \item \verb|-h|, \verb|--help| print help message;
  \item \verb|-d|, \verb|--daemon| start as daemon (detach from terminal);
  \item \verb|-p number|, \verb|--port=number| listen TCP port for incoming requests, default 8080;
  \item \verb|-s catalog|, \verb|--save=catalog| save content from ReportWare in this catalog (no default value)
  \item \verb|-t number|, \verb|--timeout=number| timeout for response from ReportWare Server, seconds (default 3)
\end{itemize}

\subsection{Deployment check}

Start \Inter{} daemon:
\begin{verbatim}
intercessor -p 8095 -s /tmp -d
\end{verbatim}

Simulate the request from script with \verb|curl|:
\begin{verbatim}
curl -x localhost:8095 -H "X-API-ReportFileName: test.xxx" \
-d 'xmlrequest=<?xml version="1.0"?>\
<RWRequest><REQUEST domain="network" service=\"ComplexReport" nocache="n" \
contact_id="1267" entity="1" filter_entity_id="1" \
clientName="ui.ent"><ROWS><ROW type="group" priority="1" ref="entity_id" \
includeascolumn="n"/><ROW type="group" priority=\"2" \
ref="advertiser_line_item_id" includeascolumn="n"/><ROW type="total"/></ROWS><COLUMNS> \
<COLUMN \
ref="advertiser_line_item_name"/><COLUMN ref="seller_imps"/><COLUMN \
ref="seller_clicks"/><COLUMN ref="seller_convs"/><COLUMN \
ref="click_rate"/><COLUMN ref="conversion_rate"/><COLUMN ref="roi"/><COLUMN \
ref="network_revenue"/><COLUMN ref="network_gross_cost"/><COLUMN \
ref="network_gross_profit"/><COLUMN ref="network_revenue_ecpm"/><COLUMN \
ref="network_gross_cost_ecpm"/><COLUMN \
ref="network_gross_profit_ecpm"/></COLUMNS><FILTERS><FILTER ref="time" \
macro="yesterday"/></FILTERS></REQUEST></RWRequest>' http://ses0316:8080/rpt
\end{verbatim}
Here is assumed that \Inter{} listen TCP port 8095 on \verb|localhost|, URL of ReportWare Server assumed as \verb|http://ses0316:8080/rpt|.

Check files \verb|/tmp/test.xxx| and \verb|/tmp/test.xxx.head|. It have to had non-zero
sizes. Content of this files should be like mentioned in section~\ref{DataView}.
If no such files, but present files like \verb|/tmp/test.xxx.DpPXyR|
and \verb|/tmp/test.xxx.head.Ua6Gw3|, then no ReportWare Server on requested URL.

Stop intercessor:
\begin{verbatim}
pkill -TERM intercessor
\end{verbatim}

\section{Implementation View\label{ImplView}}

% [This section describes the overall structure of the
% implementation model, the decomposition of the software into layers and
% subsystems in the implementation model, and any architecturally significant
% components.]

\subsection{Overview}

\Inter{} contains two functional blocks:
\begin{itemize}
  \item HTTP server; it accepts HTTP requests from Web Server scripts;
  \item HTTP client; it makes request to ReportWare Server and process response from it.
\end{itemize}

% [This subsection names and defines the various layers and
% their contents, the rules that govern the inclusion to a given layer, and the
% boundaries between layers. Include a component diagram that shows the relations
% between layers. ]

Implementation of \Inter{} based on following abstraction layers:
\begin{itemize}
  \item http---basic HTTP protocol implementation, suitable for use with C{+}{+} iostreams;
  \item StEM---C{+}{+} implementation of EDA;
  \item sockios---client and server working on top of tcp sockets;
  \item xmt---multi-thread and few core IPC techniques;
  \item BSD sockets;
  \item POSIX threads.
\end{itemize}

\subsection{Functional blocks}

\begin{figure}
\begin{center}
\scalebox{0.8}{\includegraphics{intercessor-events.pdf}}
\end{center}
\caption{Logical blocks of \Inter{}, events diagram.\label{IntercessorEventsDiagram}}
\end{figure}

\subsubsection{HTTP server\label{HTTPsrv}}

Base name of file where content of ReportWare response will be stored
passed by script in the HTTP header \verb|X-API-ReportFileName|.
This header is also used for request validation. If this header absents,
or value in this header contains symbols except alphanumeric, dot and
underscore, the request will be rejected and not translated to ReportWare.
In this case return code will be 412, ``Precondition Failed''.

\Inter{} accepts only \verb|POST| or \verb|GET| requests, otherwise return 
code will be 405, ``Method Not Allowed''.

If HTTP request looks good, \Inter{} return code 202 ``Accepted'' with zero-length
content.

\begin{table}[h]
  \centering
  \begin{tabular}{c|l|p{3in}}
   \hline
   {\fontseries{b}\selectfont Code} & {\fontseries{b}\selectfont\hfil HTTP meaning} & {\fontseries{b}\selectfont\hfil Description} \\
   \hline
   202 & Accepted & request accepted and retransmitted to ReportWare Server; ReportWare Server give positive HTTP response \\
   \hline
   405 & Method Not Allowed & HTTP command was not \verb|GET| or \verb|POST| \\
   \hline
   412 & Precondition Failed & in the HTTP request
             was missed HTTP header \verb|X-API-ReportFileName|, or value of
             HTTP header \verb|X-API-ReportFileName| contain symbol that not alphanumeric,
             dot or underscore or file(s) mentioned in \verb|X-API-ReportFileName| already exists \\
   \hline
   503 & Service Unavailable & response from ReportWare Server was negative \\
   \hline
   504 & Gateway Timeout & response from ReportWare Server
         not received within expected timeout interval (3 seconds by default) \\
   \hline
  \end{tabular}
  \caption{Summary of codes, that may be returned to script.\label{HTTPcodes}}
\end{table}

After validation of incoming request, it forwarded (as asynchronous event)
to HTTP client logical block of \Inter{} and waiting positive or negative response from it.
If response was positive (ReportWare Server give positive HTTP response within 3 seconds),
the positive response will be returned to origin of initial request. If the response was
negative, \Inter{} will return to origin of initial request the negative response of ReportWare Server as is.
If \Inter{} don't take response within 3 seconds interval, \Inter{} will return
to origin of initial request the negative response 504 ``Gateway Timeout'' (see fig.~\ref{IntercessorEventsDiagram}).

Summary of HTTP codes, that \Inter{} may return, shown in table~\ref{HTTPcodes}.

Motivation to return 202 code on success instead of ``200 Ok'', follow from
\href{http://www.w3.org/Protocols/rfc2616/rfc2616-sec10.html#sec10.2.3}{item 10.2.3 of RFC~2616} \cite{RFC2616}:

\begin{quotation}
 \noindent{\bfseries 10.2.3 202 Accepted}

 The request has been accepted for processing, but the processing
 has not been completed. The request might or might not eventually
 be acted upon, as it might be disallowed when processing actually
 takes place. There is no facility for re-sending a status code
 from an asynchronous operation such as this. 
 
 The 202 response is intentionally non-committal. Its purpose is
 to allow a server to accept a request for some other process
 (perhaps a batch-oriented process that is only run once per day)
 without requiring that the user agent's connection to the server
 persist until the process is completed. The entity returned with
 this response SHOULD include an indication of the request's current
 status and either a pointer to a status monitor or some estimate
 of when the user can expect the request to be fulfilled. 
\end{quotation}


\subsubsection{HTTP client\label{HTTPcli}}

HTTP client logical block check, that there is no files in specified catalog (option \verb|-s|)
with names like specified in \verb|X-API-ReportFileName| header (neither file with name as value from \verb|X-API-ReportFileName|
header, nor file with name as value from \verb|X-API-ReportFileName| header plus ``\verb|.head|'').
If such files was found, the request discarded.

HTTP client logical block create two files (in specified catalog, with names mentioned above) and
open its for writing. If this procedure was failed, the request discarded.

HTTP client logical block use HTTP header \verb|Host| to detect URL of ReportWare Server.
This is like all HTTP proxy do.

\Inter{} connect to ReportWare Server (using URL from \verb|Host|, as mentioned above)
and send to it the same request, as was received from Web Server script, except
\verb|X-API-ReportFileName| header and \verb|Proxy-Connection| header.

During processing response from ReportWare Server \Inter{} split HTTP headers from content,
and process content in accordance with HTTP protocol.

When \Inter{} see the code of HTTP response from ReportWare, it detect that response was
positive or negative and inform HTTP server logical block of \Inter{} via asynchronous event (see fig.~\ref{IntercessorEventsDiagram}).

\subsection{Vertical Layers}

Figure~\ref{IntercessorLayersDiagram} demonstrates decomposition of \Inter{}'s implementation into vertical layers. Layers 'complement', C++ and 'system' are external to this project.

% [For each layer, include a subsection with its name, an
% enumeration of the subsystems located in the layer, and a component diagram.]
\begin{figure}
\begin{center}
%  \input intercessor-depl.pstex_t
\scalebox{0.8}{\includegraphics{intercessor-layers.pdf}}
\end{center}
\caption{\Inter{} design, implementation layers.\label{IntercessorLayersDiagram}}
\end{figure}

\subsubsection{Intercessor}

Application is based on BSD sockets server and client, implemented in \verb|sockios|
(\ref{sockios}) and HTTP protocol implementation (\ref{httpproto}).

\subsubsection{HTTP protocol\label{httpproto}}

Due to HTTP protocol is text-based, it may be effectively implemented
on top of C++ \verb|istream|/\verb|ostream|.

\subsubsection{EDA implementation\label{StEM}}

EDA implemented in \verb|StEM| library (part of \cite{ComplementProject}). 

\subsubsection{BSD TCP sockets, client and server\label{sockios}}

BSD sockets server and client implemented in \verb|sockios| library, part of
\cite{ComplementProject}. It is based on
multi-threading server concept and use \verb|xmt| (\ref{xmt}) library.

\subsubsection{Threads and IPC\label{xmt}}

Deal with threads, synchronization primitives and IPC methods
is encapsulated in \verb|xmt| library, part of \cite{ComplementProject}.

\subsubsection{System}

Underling level is ordinary POSIX-like OE with POSIX threads and BSD
sockets. Any modern Linux acceptable.

\section{Data View\label{DataView}}

Raw HTTP request from Web Server script (every line is ended with CR/LF pair).
\begin{verbatim}
POST http://localhost:8090/test.php HTTP/1.1
Host: localhost:8090
X-API-ReportFileName: test.xxx
Content-Length: 890

xmlrequest=<?xml version="1.0"?>
<RWRequest><REQUEST domain="network" service="ComplexReport" nocache="n" 
contact_id="1267" entity="1" filter_entity_id="1"
clientName="ui.ent"><ROWS><ROW type="group" priority="1" ref="entity_id"
includeascolumn="n"/><ROW type="group" priority="2"
ref="advertiser_line_item_id" includeascolumn="n"/><ROW type="total"/></ROWS>
<COLUMNS><COLUMN
ref="advertiser_line_item_name"/><COLUMN ref="seller_imps"/><COLUMN
ref="seller_clicks"/><COLUMN ref="seller_convs"/><COLUMN
ref="click_rate"/><COLUMN ref="conversion_rate"/><COLUMN ref="roi"/><COLUMN
ref="network_revenue"/><COLUMN ref="network_gross_cost"/><COLUMN
ref="network_gross_profit"/><COLUMN ref="network_revenue_ecpm"/><COLUMN
ref="network_gross_cost_ecpm"/><COLUMN
ref="network_gross_profit_ecpm"/></COLUMNS><FILTERS><FILTER ref="time"
macro="yesterday"/></FILTERS></REQUEST></RWRequest>
\end{verbatim}

Content of file with headers from ReportWare Server response (\verb|test.xxx.head|, for
request above):
\begin{verbatim}
HTTP/1.1 200 Ok
Server: Apache-Coyote/1.1
Content-Type: text/html;charset=ISO-8859-1
Content-Length: 8132
Date: Sun, 18 Feb 2007 16:22:04 GMT

\end{verbatim}

Content of file with data from ReportWare Server response (\verb|test.xxx|, for
request above):
\begin{verbatim}
<?xml version="1.0" encoding="ISO-8859-1"?>
<!DOCTYPE html PUBLIC "-//W3C//DTD XHTML 1.0 Strict//EN"
   "http://www.w3.org/TR/xhtml1/DTD/xhtml1-strict.dtd">

<html xmlns="http://www.w3.org/1999/xhtml" xml:lang="en" lang="en">
    <head>
    <title>Apache Tomcat/5.5.20</title>
    <style type="text/css">
    /*<![CDATA[*/
      body {
          color: #000000;
...

    </tr>
</table>

</body>
</html>
\end{verbatim}


% [A description of the persistent data storage perspective of
% the system. This section is optional if there is little or no persistent data,
% or the translation between the Design Model and the Data Model is trivial.]

\section{Size and Performance\label{SizePerf}}

% [A description of the major dimensioning characteristics of
% the software that impact the architecture, as well as the target performance
% constraints.]

At present time no sure information about size of indicative requests and
responses.

The perforamance aspects unknown too: intensity of requests unfortunately
not evident too.

\section{Quality\label{Quality}}

% [A description of how the software architecture contributes
% to all capabilities (other than functionality) of the system: extensibility,
% reliability, portability, and so on. If these characteristics have special
% significance, for example safety, security or privacy implications, they should
% be clearly delineated.]

The unit tests for all components involved ('check' procedure, section~\ref{DeplCheck})
allow to detect problem on early stage. But this tests don't make massive testing,
testing with real scripts and real ReportWare Server\footnote{Test with real
ReportWare Server available, but there are difficulties with rest of test infrastructure.}.

\Inter{} may be used for requests logging, signaling and other notifications, including
notifications about some events. This functionality
may be added relatively easy, because of asynchronous nature of \Inter{} architecture.

% reliability

% No requirements for reliability?!

% portability

No special requirements for portability. Expected, that code of \Inter{} and related
components will be compiled and run under any POSIX-compliant architecture.

% safety

The aspects of authentication, validation of data, safety of stored files should be taken into
account during administration of the OE, and not considered within scope of \Inter{} server.

\subsection{Known restrictions and bugs}

At present time known following potential problems:
\begin{itemize}
  \item HTTP header \verb|Keep-Alive| is not taken into account;
  \item persistent connection is not supported; this is normal for connection with
        WebServer script, but may be treated as drawback in connection with
        ReportWare server (extra load on ReportWare server);
  \item HTTP protocol is announced as HTTP 1.1, but not all protocol issues were implemented,
        and switching to HTTP 1.0 was not considered;
  \item \Inter{} when become daemon, don't save own pid in any file; this may be not very convenient
        for administration and monitoring.
\end{itemize}

\end{document}
