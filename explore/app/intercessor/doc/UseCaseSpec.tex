% -*- LaTeX -*-
% $Id: UseCaseSpec.tex,v 1.1 2002/11/04 10:54:23 ptr Exp $

\documentclass[a4paper,twoside]{article}

\usepackage{rup}

\title{Use-Case-Realization Specification: ``Use-Case Name''}

\project{Virtual Manual}
% \RevHistory{dd/mmmm/yyyy & x.x & & \cr \hline}

\begin{document}

\maketitle

\section{Introduction}

[The introduction of the Use-Case Realization Specification
should provide an overview of the entire document. It should include the
purpose, scope, definitions, acronyms, abbreviations, references, and overview
of this Use-Case Realization Specification.]

[Note: This document template assumes that the use-case
realization is partly described within a Rational Rose model; this means that
the use case's name and brief description is within the Rose model, and that
this document should be linked as an external file to the use case. This
document should contain additional properties of the use-case realization that
are not in the Rose model.]

\subsection{Purpose}

[Specify the purpose of this Use-Case Realization Specification]

\subsection{Scope}

[A brief description of the scope of this Use-Case
Realization Specification; what Use Case model(s) it is associated with,
and anything else that is affected or influenced by this document.]

\subsection{Definitions, Acronyms and Abbreviations}

[This subsection should provide the definitions of all terms,
acronyms, and abbreviations required to properly interpret the Use-Case
Realization Specification. This information may be provided by
reference to the project Glossary.]

\subsection{References}

[This subsection should provide a complete list of all
documents referenced elsewhere in the Use-Case Realization Specification.
Each document should be identified by title,
report number (if applicable), date, and publishing organization. Specify the sources from which the
references can be obtained. This information may be provided by reference to an
appendix or to another document.]

\subsection{Overview}

[This subsection should describe what the rest of the Use-Case
Realization Specification contains and explain how the document is
organized.]

\section{Flow of Events Design}

[A textual description of how the use case is realized in
terms of collaborating objects. Its main purpose is to summarize the diagrams
connected to the use case and to explain how they are related.]

\section{Derived Requirements}

[A textual description that collects all requirements, such
as non-functional requirements, on the use-case realizations that are not
considered in the design model, but that need to be taken care of during
implementation.]

\end{document}
